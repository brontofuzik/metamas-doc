%%%%%%%%%%%%%%%%%%%%%%%%%%%%%%%%%%%%%%%%%%%%%%%%%%%%%%%%%%%%%%%%%%%%%%%%%%%%%%%%
%% Title (en): Multiagent Systems and Organizations                           %%
%% Title (cs): Multiagentní systémy a organizace                              %%
%%                                                                            %%
%% Author: Bc. Lukáš Kúdela                                                   %%
%% Supervisor: Prof. RNDr. Petr Štěpánek, DrSc.                               %%
%%                                                                            %%
%% Academic year: 2011/2012                                                   %%
%%%%%%%%%%%%%%%%%%%%%%%%%%%%%%%%%%%%%%%%%%%%%%%%%%%%%%%%%%%%%%%%%%%%%%%%%%%%%%%%

\chapter{Agents and Multiagent Systems}

% Purpose of this chapter - introduce agents and MASs
The purpose of this chapter is to introduce agents and multiagent systems (MASs).
The introduction is deliberately kept as short as possible - sticking to the most fundamental characteristics.
Nevertheless, understanding of these is sufficient to comprehend ideas presented in this thesis (ALT: text).
It is not possible to do justice to the field of agents and MAS anyway; the wide applicability of agents and MASs inevitably causes the field of their study to be vast.

% Agents and MAS - referecnes
A reader looking for a crash course on MAS can try \cite{Wooldridge02} and \cite{Wooldridge95}; a reader seeking a more in-depth (yet still general) treatise of MAS will not go wrong with any of the well-established works on the subject: \cite{Wooldridge09} and \cite{Weiss99}.
Alternatively, in \cite{Shoham08} the authors take an algorithmic, game-theoretic and logical\footnote{As in mathematical-logical, not rational.} approach to MASs.
To our knowledge, \cite{Kubik04} or \cite{Kubik00} is the most complete coverage of MAS in Czech, while \cite{Kubik03} is an overview by the same author.

%%%%%%%%%%%%%%%%%%%%%%%%%%%%%%%%%%%%%%%%%%%%%%%%%%%%%%%%%%%%%%%%%%%%%%%%%%%%%%%%
\section{Agents}

% No single universally accepted definition of agent exists
Unfortunately, no single universally accepted definition of agent exists in the agent community.
However, this does not seem to be a problem at all; despite the lack of agreement on terminological details, many researchers are coming up with interesting agent theories and numerous practitioners are developing useful agent applications.
It is still advantageous that at least \textit{some} definition of agent exists though - if for nothing else then to protect it from being misused.

% Two usages of the term `agent'
Two usages of the term \textit{agent} can be recognized: the first is weaker and does not appear to be disputed (ALT: contended), and the second is stronger and generates more discussion in the community.
In this thesis, it is sufficient to use the weaker notion of agency.

% Agent - definition: situatedness (sensors, actuators), autonomy, reactivity, proactivity and social skills
An \textit{agent} is a computer system (ALT: computational entity) that is situated in some environment (physical or software) and that is capable of autonomous action in this environment in order to meet its design objectives \cite{Wooldridge02}.
Being situated in an environment means that the agent can \textit{sense} or \textit{perceive} it (through its \textit{sensors}) and \textit{act} upon it (through its \textit{actuators}).
An agent can be \textit{reactive} (capable of responding in a timely fashion to changes that occur in the environment \cite{Wooldridge95}) and/or \textit{proactive} (able to exhibit goal-directed behaviour by taking the initiative \cite{Woldridge95}).
Finally, an agent in a MAS is required to have \textit{social skills} (skills facilitating interaction and communication with other agents).

%%%%%%%%%%%%%%%%%%%%%%%%%%%%%%%%%%%%%%%%%%%%%%%%%%%%%%%%%%%%%%%%%%%%%%%%%%%%%%%%
\section{Multiagent Systems}

Multiagent (alernatively multi-agent) system is a set of interacting or interdependent components - agents - forming an integrated whole - a society of agents.
The interaction can be direct (via messages) and indirect (via environment).