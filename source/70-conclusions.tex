%%%%%%%%%%%%%%%%%%%%%%%%%%%%%%%%%%%%%%%%%%%%%%%%%%%%%%%%%%%%%%%%%%%%%%%%%%%%%%%%
%% MASTER'S THESIS                                                            %%
%%                                                                            %% 
%% Title (en): Multi-Agent Systems and Organizations                          %%
%% Title (cs): Multiagentní systémy a organizace                              %%
%%                                                                            %%
%% Author: Bc. Lukáš Kúdela                                                   %%
%% Supervisor: Prof. RNDr. Petr Štěpánek, DrSc.                               %%
%%                                                                            %%
%% Academic year: 2011/2012                                                   %%
%%%%%%%%%%%%%%%%%%%%%%%%%%%%%%%%%%%%%%%%%%%%%%%%%%%%%%%%%%%%%%%%%%%%%%%%%%%%%%%%

\chapwithtoc{Conclusions and Future Work}

% Conclusion %%%%%%%%%%%%%%%%%%%%%%%%%%%%%%%%%%%%%%%%%%%%%%%%%%%%%%%%%%%%%%%%%%%

% Thesis objectives
The objectives of this thesis were to investigate some of the metamodel-based approaches to modelling organizations in multi-agent systems, to propose a new organizational metamodel inspired the existing ones, to implement this metamodel in a free and open-source mainstream general-purpose agent platform, and to demonstrate its use with a number of examples. 

% 1. Investigate some of the metamodel-based approaches to modelling organizations in multi-agent systems - Aalaadin, O&P, PIM4Agents and powerJade
We have investigated a total of four organizational metamodels and have drawn inspiration from all of them.
We have observed that they all more or less agree on the core concepts (organization, role), but offer different views of the auxiliary concepts.

% 2. Propose a new organizational metamodel inspired the existing ones - Thespian
Inspired by the existing ones, we have proposed a new organizational metamodel---\textit{Thespian}.
\textit{Thespian} is a fusion of investigated metamodels and some original ideas, for example, the mechanism of subscribing to organization events, their publishing and subsequent handling.
\textit{Thespian} was designed to strike a balance between expresiveness and simplicity; it provides enough concepts for modelling real world organizations, but at the same avoids introducing modelling constructs peripheral to this domain.
In other words, in \textit{Thespian} we have made an effort to provide a tool that does just one job---modelling organizations---and does it well.

% 3. Implement this metamodel in a mainstream general-purpose agent platform - Thespian4Jade
To verify the applicability of \textit{Thespian}, we have implemented it as an extension for \textit{Jade} (the most popular free and open-source general-purpose agent platform) called \textit{Thespian4Jade}.
To our knowledge, \textit{Thespian4Jade}---being an actually implemented\footnote{As opposed to just specified.} organizational extension of an existing general-purpose agent platform---is an unprecedented\footnote{And at the time of writing this thesis also unique.} effort.

% 4. Demonstrate its usage with a number of examples
In order to demonstrate the use of \textit{Thespian4Jade}, we have modelled three example organizations ranging from the simplest possible (remote function invocation) through a sightly more involved (arithmetic expression evaluation) to a real-world organization (sealed-bid auction).
With these examples we have highlighted the separation of behaviour acquired through a role from the behaviour inherent to a player.
This decoupling of innate and acquired behaviour has one particularly appealing consequence---an organization can be designed and developed independently from the players who will participate in it. 

% Future Work %%%%%%%%%%%%%%%%%%%%%%%%%%%%%%%%%%%%%%%%%%%%%%%%%%%%%%%%%%%%%%%%%%

Although we are confident \textit{Thespian} is a sufficiently expressive metamodel and \textit{Thespian4Jade} is a useful \textit{Jade} extension, there are a few places where we made certain assumptions and imposed some restrictions.
This way, we could do without some features that would otherwise have been indispensable.
These features are outlined here as the strongest candidates for any future work.

% Organization manager
It is assumed that a concrete organization in which a player wants to enact a role already exists.
It has to be specified at compile-time, and it is created at MAS start-up.
Undoubtedly, a more flexible approach, where a player could request that an organization of certain type be created, would be a step in the right direction.
For this to work, a special agent---\textit{Organization manager}---would have to exist capable of creating organizations on demand.

% Yellow pages for players & organization-initiated role enactment.
Currently, only a player can initiate role enactment.
% Reality
This is not an accurate reflection of reality.
In the real world, not only a job-seeker can initiate the recruitment process; a company can take the initiative by selecting candidates for a position from a pool of all job-seekers and making them an offer.
% Thespian
Two features would have to be added to \textit{Thespian} in order to support this process: \textit{yellow pages for players} that would enable an organization to search for existing players with certain capabilities and \textit{organization-initiated role enactment}.

% Responsibilities fulfilling monitoring & organization-initiated role deactment
At present, role deactment can only by initiated by a player.
% Reality
Again, this is not true in the real world.
In reality, an employee quitting a job is not the only way for them to stop being employed; a company can fire an employee if they fail to fulfil their position's responsibilities.
% Thespian
In order for \textit{Thespian} to support this procedure, two features would have to be added: \textit{responsibilities fulfilling monitoring} that would enable an organization to keep track of its players' performance and \textit{organization-initiated role deactment}.

% Holonic MAS
Another restriction is that only a player can play a role.
At first sight, this does not seem like a too big a restriction, if it is a restriction at all.
However, an organization could be viewed as a player as well, capable of more than its individual members put together.
Therefore, it makes sense to consider the possibility of organizations playing roles in other organizations.
Such an organization is a \textit{holon} (simultaneously a whole and a part) in a \textit{holonic MAS}.
\textit{Thespian}, and \textit{Thespian4Jade} in particular, are well prepared to accommodate this feature, with the organizations already being agentified.