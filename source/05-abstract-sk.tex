%%%%%%%%%%%%%%%%%%%%%%%%%%%%%%%%%%%%%%%%%%%%%%%%%%%%%%%%%%%%%%%%%%%%%%%%%%%%%%%%
%% MASTER'S THESIS                                                            %%
%%                                                                            %% 
%% Title (en): Multi-Agent Systems and Organizations                          %%
%% Title (cs): Multiagentní systémy a organizace                              %%
%%                                                                            %%
%% Author: Bc. Lukáš Kúdela                                                   %%
%% Supervisor: Prof. RNDr. Petr Štěpánek, DrSc.                               %%
%%                                                                            %%
%% Academic year: 2011/2012                                                   %%
%%%%%%%%%%%%%%%%%%%%%%%%%%%%%%%%%%%%%%%%%%%%%%%%%%%%%%%%%%%%%%%%%%%%%%%%%%%%%%%%

% Temporarily select the Slovak language (with French spacing).
\selectlanguage{slovak}
\frenchspacing

Názov: Multiagentové systémy a organizácie\\
Autor: Bc. Lukáš Kúdela\\
E-mailová adresa autora: \url{lukas.kudela@gmail.com}\\
Katedra: Katedra teoretické informatiky a matematické logiky\\
Veducí práce: Prof. RNDr. Petr Štěpánek, DrSc.\\
E-mailová adresa vedúceho: \url{petr.stepanek@mff.cuni.cz}\\

% Motivácia
Abstrakt: Multiagentové systémy (MAS) sa ukazujú ak sľubná paradigma pre konceptualizáciu, návrh a implementáciu rozsiahlych heterogénnych softvérových systémov.
Hlavná výhoda pozerania sa na komponenty v takých systémoch ako na autonómne agenty spočíva v tom, že ako agenty sú schopné flexibilnej samoorganizácie, namiesto toho, aby boli rigidne organizované systémovým architektom.
Avšak, samoorganizácia je ako evolúcia -- vyžaduje veľa času a výsledky nie sú zaručené.
Systémový architekt má často predstavu o tom, ako by sa mali agenti organizovať -- aké typy organizácií by mali vytvárať.
% Stanovenie problému
V našej práci sme sa pokúsili vyriešiť problém modelovania organizácií a ich rolí v MAS, nezávisle na konkrétnej agentovej platforme, na ktorej MAS napokon pobeží.
% Prístup
V prvom rade sme navrhli metamodel na popis platformovo nezávislých organizačných modelov.
Ďalej sme navrhnutý model implementovali pre agentovú platformu Jade ako modul rozširujúci tento framework.
Napokon sme predviedli použitie nášho modulu vymodelovaním troch konkrétnych organizácií: vzdialené vyvolanie funkcie, vyhodnotenie aritemetického výrazu a aukcia obálkovou metódou. 
% Výsledky
Naša práca ukazuje ako oddeliť rolou nadobudnuté chovanie od chovania, ktoré je neoddeliteľnou súčasťou agenta.
% Závery
Táto separácie umožňuje, aby boli organizácie vyvýjané nezávisle od agentov, ktoré v nich budú participovať a tým uľahčuje vývoj tzv. otvorených systémov.

Kľúčové slová: multiagentové systémy, organizácie, role, metamodel

% Select the English language (with English spacing) again.
\selectlanguage{english}
\nonfrenchspacing