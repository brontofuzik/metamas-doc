%%%%%%%%%%%%%%%%%%%%%%%%%%%%%%%%%%%%%%%%%%%%%%%%%%%%%%%%%%%%%%%%%%%%%%%%%%%%%%%%
%% MASTER'S THESIS                                                            %%
%%                                                                            %% 
%% Title (en): Multi-Agent Systems and Organizations                          %%
%% Title (cs): Multiagentní systémy a organizace                              %%
%%                                                                            %%
%% Author: Bc. Lukáš Kúdela                                                   %%
%% Supervisor: Prof. RNDr. Petr Štěpánek, DrSc.                               %%
%%                                                                            %%
%% Academic year: 2011/2012                                                   %%
%%%%%%%%%%%%%%%%%%%%%%%%%%%%%%%%%%%%%%%%%%%%%%%%%%%%%%%%%%%%%%%%%%%%%%%%%%%%%%%%

\chapter{Organizations in Multi-Agent Systems}

% Quote
\begin{flushright}
\textit{``The achievements of an organization are the results of the combined effort of each individual.''}\\
\textit{--- Vince Lombardi, American football coach}
\end{flushright}

% Chapter intro
In this chapter, we discuss the motivation for introducing organizations to MASs, introduce two conceptions of a MAS and talk about some key organizational concepts for the first time in an informal manner.

% Decomposable problem with independent subproblems
Consider a problem with the following properties:
\begin{itemize}
	\item \textit{decomposability} --- It can easily be decomposed into well-defined subproblems. These subproblems can either be decomposable themselves or atomic.
	Note that the subproblems do not necessarily have to resemble the original problem or each other\footnote{A characteristic necessary for divide and conquer algorithms.}.
	\item \textit{subproblem independence} --- The subproblems can be solved more or less independently and therefore, providing enough computational entities are available, concurrently.
	Note that this is a quantitative characteristic, not a categorical one.
\end{itemize}

% Solving such a problem with a MAS
Such a problem can be solved by first pondering how a society of humans (viewed as intelligent autonomous computational entities) would go about solving it, then modelling this society as a MAS and finally running the system.
However there is an issue we must address---organizational structure of human societies.

% Organizational structure of human societies
In all human societies but the most primitive ones, various types of \textit{organizations}\comments{FO} emerge to facilitate cooperation among their members.
In these organization types \textit{roles}\comments{FO} appear and \textit{interaction protocols}\comments{FO} governing their interaction crystallize among them.
We are facing the challenge of carrying these \textit{organizational concepts} over to the realm of MASs.

% No standardized way to impose organizational structure upon MASs
Unfortunately, there is no standardized way to impose organizational structure upon MASs.
It should come as no surprise to anybody who is familiar with MASs that no single, precise and universally accepted notions of organization, role or interaction protocol currently exist among researchers.
In a plain vanilla MAS, every agent can (in principle) talk to any other agent, regardless of whether this is desirable or even allowed in the society being modelled by the MAS.

% ACMAS vs. OCMAS
In the next two sections, we will introduce two ways of looking at MASs:
\begin{itemize}
	\item \textit{agent-centric} --- focused on the structure of individual agents (the traditional viewpoint), and
	\item \textit{organization-centric} --- focused on the structure of agent societies (a novel perspective).
\end{itemize}

%%%%%%%%%%%%%%%%%%%%%%%%%%%%%%%%%%%%%%%%%%%%%%%%%%%%%%%%%%%%%%%%%%%%%%%%%%%%%%%%
\section{Agent-Centric Multi-Agent Systems}

% ACMAS - architecture of individual agents
An \textit{agent-centric multi-agent system} (ACMAS) is the classical conception of a MAS.
It focuses on the architecture of individual agents, being oblivious to the structure of their society.

% ACMAS - characteristics
An ACMAS has the following characteristics \cite{Ferber03}:
\begin{itemize}
	\item Every agent has a public \textit{agent identifier}\footnote{\textit{Agent identifier} (AID) is a name that identifies (that is, labels the identity of) a unique agent.} and it can be addressed with it.
	\item An agent can communicate with any other agent\footnote{Of course, the agent needs to know the other agent's AID, but since these are public, this is not an obstacle.}.
	\item An agent provides a set of services, which are available to every other agent in the system.
	\item It is the responsibility of each agent to constrain its accessibility and the accessibility of its services to other agents.
	\item It is the responsibility of each agent to define its relations, contracts, etc. with other agents.
\end{itemize}

Perhaps ironically, the absolute freedom of interaction in an ACMAS is the cause of many of its shortcomings \cite{Ferber03}:
\begin{itemize}
	\item Predicting the behaviour of the whole system from the behaviour of its constituent components is extremely difficult, if not downright impossible, due to high probability of emergent behaviour.
	\item Because there is no implicit security management, it is easy for a malicious agent to unknowingly misuse or even intentionally abuse the system.
	\item It is not possible to apply the principles of \textit{modular design}. Agents cannot be grouped into modules with different visibilities to the outside world (public vs. private) at design-time, let alone at run-time.
	\item It is not possible to pursue the \textit{framework approach}. There is only one framework---the agent platform itself---and it is impossible to define sub-frameworks with specific interactions.
\end{itemize}

%%%%%%%%%%%%%%%%%%%%%%%%%%%%%%%%%%%%%%%%%%%%%%%%%%%%%%%%%%%%%%%%%%%%%%%%%%%%%%%%
\section{Organization-Centric Multi-Agent Systems}

% OCMAS - structure of agent society (agent social structure)
An \textit{organization-centric multi-agent system} (OCMAS) is the modern conception of MAS proposed in \cite{Ferber03}.
It focuses on the structure of the agent society, paying no attention to the architecture of individual agents.

% OCMAS - organizational level vs. agent level
Organizations provide a natural way of describing structure of a MAS and interactions among its constituent agents.
This description is situated on the \textit{organization level} of an OCMAS, the level above the \textit{agent level}, which is the only level considered in an ACMAS.
The organizational level contains abstract representations of the concrete organizations occurring on the agent level; in particular, it specifies the types of organizations, their roles and interaction protocols among them, which should occur on the agent level.

% OCMAS - characteristics
The following are the characteristics of an OCMAS \cite{Ferber03}:
\begin{itemize}
	\item The organizational level imposes social structure and patterns of interaction upon agents, but does not prescribe how agents should behave; it merely demarcates the space within which the agents can express their individuality.
	\item The organizational level does not place constraints on the architectures of the agents; deliberative as well as reactive agent can take part in an organization as longs as they behave in an expected way.
	\item The organizations provide a way to partition a MAS into bounded contexts of interaction.
	Whereas the structure of an organization is known to its members and they are able to interact with one another, it is opaque to the non-members whose interaction with the organization (its members) is limited.
\end{itemize}

\section{Organizational Concepts}

% Organization
An \textit{organization} is a a structured group of agents, which imposes rules on the behaviour and mutual interaction of its members. 
These rules are imposed by roles and interaction protocols defined in the organization.

% Role
A \textit{role} is an interface between an organization and its member; the organization interacts with its members through their roles.
Is is also an interface between the organization members themselves; the members interact with each other via the roles they play.
A role always exists and operates within the context of its defining organization.

% Competence & responsibility
When playing a role, a player is entitled to exercise the role's competences but also obliged to fulfil its responsibilities.
A \textit{competence} is an operation the role's player \textit{can} perform as a result of playing that role.
A \textit{responsibility} is an operation the role's player \textit{has to} perform as a consequence of playing that role.

% Interaction protocol
An \textit{interaction protocol} is a institutionalized pattern of interaction\footnote{In this thesis, the only kind of interaction we consider is communication. Therefore, we will use the terms \textit{interaction} and \textit{communication} interchangeably.} between certain roles in an organization.
It defines by intension a set of possible communication scenarios between the players of these roles.
In the context of a protocol, the participating roles (or their players) are called \textit{parties}.