%%%%%%%%%%%%%%%%%%%%%%%%%%%%%%%%%%%%%%%%%%%%%%%%%%%%%%%%%%%%%%%%%%%%%%%%%%%%%%%%
%% MASTER'S THESIS                                                            %%
%%                                                                            %% 
%% Title (en): Multi-Agent Systems and Organizations                          %%
%% Title (cs): Multiagentní systémy a organizace                              %%
%%                                                                            %%
%% Author: Bc. Lukáš Kúdela                                                   %%
%% Supervisor: Prof. RNDr. Petr Štěpánek, DrSc.                               %%
%%                                                                            %%
%% Academic year: 2011/2012                                                   %%
%%%%%%%%%%%%%%%%%%%%%%%%%%%%%%%%%%%%%%%%%%%%%%%%%%%%%%%%%%%%%%%%%%%%%%%%%%%%%%%%

Title: Multi-Agent Systems and Organizations\\
Author: Bc. Lukáš Kúdela\\
Author's e-mail address: \url{lukas.kudela@gmail.com}\\
Department: Department of Theoretical Computer Science and Mathematical Logic\\
Thesis Supervisor: Prof. RNDr. Petr Štěpánek, DrSc.\\
Supervisor's e-mail address: \url{petr.stepanek@mff.cuni.cz}\\

% Motivation - Why do we care about the problem and the results?
Abstract: Multi-agent systems (MAS) are emerging as a promising paradigm for conceptualizing, designing and implementing large-scale heterogeneous software systems.
The key advantage of looking at components in such systems as autonomous agents is that as agents they are capable of flexible self-organization, instead of being rigidly organized by the system's architect.
However, self-organization is like evolution---it takes a lot of time and the results are not guaranteed.
More often than not, the system's architect has an idea about how the agents should organize themselves---what types of organizations they should form.
% Problem statement - What problem are we trying to solve?
In our work, we tried to solve the problem of modelling organizations and their roles in a MAS, independent on the particular agent platform on which the MAS will eventually run.
% Approach - How did we go about solving or making progress on the problem?
First and foremost, we have proposed a metamodel for expressing platform-independent organization models.
Furthermore, we have implemented the proposed metamodel for the Jade agent platform as a module extending this framework.
Finally, we have demonstrated the use of our module by modelling three specific organizations: remote function invocation, arithmetic expression evaluation and sealed-bid auction.
% Results - What's the answer?
Our work shows how to separate the behaviour acquired through a role from the behaviour intrinsic to an agent. 
% Conclusions - What are the implications of our answer?   
This separation enables organizations to be developed independently of the agents that will participate in them, thus facilitating the development of the so-called open systems.

Keywords: multi-agent systems, organizations, roles, metamodel