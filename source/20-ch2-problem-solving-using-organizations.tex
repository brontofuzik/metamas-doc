%%%%%%%%%%%%%%%%%%%%%%%%%%%%%%%%%%%%%%%%%%%%%%%%%%%%%%%%%%%%%%%%%%%%%%%%%%%%%%%%
%% Title (en): Multiagent Systems and Organizations                           %%
%% Title (cs): Multiagentní systémy a organizace                              %%
%%                                                                            %%
%% Author: Bc. Lukáš Kúdela                                                   %%
%% Supervisor: Prof. RNDr. Petr Štěpánek, DrSc.                               %%
%%                                                                            %%
%% Academic year: 2011/2012                                                   %%
%%%%%%%%%%%%%%%%%%%%%%%%%%%%%%%%%%%%%%%%%%%%%%%%%%%%%%%%%%%%%%%%%%%%%%%%%%%%%%%%

\chapter{Problem Solving using Organizations}

% Quote
\begin{flushright}
\textit{``The achievements of an organization are the results of the combined effort of each individual.''}\\
\textit{-- Vince Lombardi, American football coach}
\end{flushright}

% Decomposable problem with independent subproblems
Consider a problem with the following properties:
\begin{itemize}
	\item \textbf{Decomposability} It can easily be decomposed into well-defined subproblems. These subproblems can either be decomposable themselves or atomic.
	Note that the subproblems do not necessarily have to resemble the original problem or each other\footnote{A problem characteristic necessary for divide and conquer algorithms.}. 
	\item \textbf{Subproblem independence} The subproblems can be solved more or less independently and therefore, providing enough computational entities are available, simultaneously (ALT: concurrently).
	Note that this is a continuous characteristic, not a categorical (yes or no) one.
\end{itemize}

% Solving such a problem with a MAS
Such a problem can be solved by first pondering how a society of humans (intelligent autonomous computational entities) would go about solving it, then modelling this society as a multiagent system and finally running the MAS.
However there is an issue we must address - organizational structure of human societies.

%%%%%%%%%%%%%%%%%%%%%%%%%%%%%%%%%%%%%%%%%%%%%%%%%%%%%%%%%%%%%%%%%%%%%%%%%%%%%%%%
\section{Organizational Structure of Human Societies}

% Organizational structure of human societies
In all human societies but the most primitive ones, various types of \textit{organizations} emerge to facilitate cooperation of their members.
In these organization types \textit{roles} appear and \textit{interaction protocols} governing their interaction crystallize among them.
We are facing the challenge of carrying these \textit{organizational concepts} over to the realm of MAS.
It should come as no surprise to anybody who is familiar with the field of MAS that no a) single, b) precise (ALT: exact) and c) widely agreed upon (ALT: universally accepted) notions of organization, role or interaction protocol currently exist within it.

% Plain vanilla MAS - every agent can talk to any other agent
In plain vanilla MAS, every agent can (in principle) talk to any other agent, regardless of whether this is desirable or even allowed in the modelled society.
There is no standardized way to impose organizational structure upon a MAS (ALT: MASs). 
In the next section, we will introduce a new type of MAS, focused on the society of agents and its structure as opposed to individual agents, which seem to be the focus of traditional MAS.

%%%%%%%%%%%%%%%%%%%%%%%%%%%%%%%%%%%%%%%%%%%%%%%%%%%%%%%%%%%%%%%%%%%%%%%%%%%%%%%%
\section{Agent-Centric MAS (ACMAS) vs. Organization-Centric MAS (OCMAS)}

% ACMAS vs. OCMAS


% Organizational concept: organization
An \textit{organization} is a an (organized?) group of agents, which imposes rules on the mutual interaction of its members. 
These rules are imposed by (ALT: via) roles defined in the organization.

% Organizational concept: role
A \textit{role} is an interface between an organization and its members; the organization interacts with its members via their roles.
Is is also an interface between the organization members themselves; the members interact with each other via the roles they play.

% Organizational concept: competence and responsibility
A role is a collection of \textit{competences} - operations the player can do at its own will when playing that role - and \textit{responsibilities} - operations the player must do at the role's will when playing that role.
A role always exists within the context of its defining organization.
% TODO More precisely, a role can only be defined in the context of an organization type and a position (role instance) always exists in the context of

% Interaction protocol and party


%%%%%%%%%%%%%%%%%%%%%%%%%%%%%%%%%%%%%%%%%%%%%%%%%%%%%%%%%%%%%%%%%%%%%%%%%%%%%%%%
\section{Organizational Structure of MASs}
