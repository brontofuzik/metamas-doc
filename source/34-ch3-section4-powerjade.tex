%%%%%%%%%%%%%%%%%%%%%%%%%%%%%%%%%%%%%%%%%%%%%%%%%%%%%%%%%%%%%%%%%%%%%%%%%%%%%%%%
%% powerJade                                                                  %%
%%%%%%%%%%%%%%%%%%%%%%%%%%%%%%%%%%%%%%%%%%%%%%%%%%%%%%%%%%%%%%%%%%%%%%%%%%%%%%%%
\section{powerJade}

% powerJade - references
This section introduces the powerJade\footnote{Intentionally uncapitalized.} metamodel described in \cite{Baldoni08a}, \cite{Baldoni08b}, \cite{Baldoni09} and \cite{Baldoni10}.
The presented overview is due to \cite{Hahn07b} in particular.

% powerJade - authors
powerJade has been put forward in 2008 by Matteo Baldoni, Guido Boella and their colleagues from University of Turin in Turin, Italy.

%% Summary %%%%%%%%%%%%%%%%%%%%%%%%%%%%%%%%%%%%%%%%%%%%%%%%%%%%%%%%%%%%%%%%%%%%%

% powerJade inspired by powerJava
powerJade is a platform independent metamodel inspired by the authors' previous work on powerJava - the extension of the Java programming language with an explicit role construct based on an ontological analysis of roles (see \cite{Baldoni05}, \cite{Baldoni06a}, \cite{Baldoni06b} and \cite{Baldoni07}).

% Definitions (organization, role)
An organization (or, more generally, an institution) belongs to the social reality and it can only be interacted with via the roles it contains \cite{Boella06}.
It is not an object (as in OOP) that could be manipulated from outside.
The (modelling?) concept of an \textit{organization} is not only advantageous when modelling problem domains including organizations/institutions of some kind.
Indeed, we can view every object (as in OOP) as an organization/institution offering different ways of interacting with it, each of which is represented by a different role.

%%%%%%%%%%%%%%%%%%%%%%%%%%%%%%%%%%%%%%%%%%%%%%%%%%%%%%%%%%%%%%%%%%%%%%%%%%%%%%%%
\subsection*{Organization Structure Metamodel}
% ALT: Ontological Model of an Organization, Organization Ontology

An ontological analysis of roles in \cite{Boella04} yields the following properties of roles:
\begin{itemize}
	\item \textit{Foundation} - a role instance is always associated with an instance of the organization class to which is belongs and with a player instance.
	\item \textit{Definitional dependence} - the role definition depends on the organizations it belongs to.
	\item \textit{Institutional powers} - the role operations (called \textit{powers}) have access to the state of the organization and other roles of the organization.
	\item \textit{Prerequisities} - to be granted (and play) a role, the player must be able to perform operations (called \textit{requirements}) which can be requested while it plays the role.
\end{itemize}

% Ontological status of organizations and roles compared to agents
The metamodel in \cite{Boella04} focuses on the organization structure.
The ontological status of organizations and roles \cite{Boella04} does not differ completely from that of agents (AOP) or objects (OOP).
On one hand, roles do not exist as independent entities (like objects in OOP), since they are necessarily linked to organizations.
Additionally, organizations and roles are not autonomous and act via players.
On the other hand, organizations and roles, like agents, are descriptions of complex behaviour.
In the real world, organizations are considered legal entities; they can even act like agents, although via a representative role.
Since they share some properties with agents, they can be modelled using similar primitives.

%%%%%%%%%%%%%%%%%%%%%%%%%%%%%%%%%%%%%%%%%%%%%%%%%%%%%%%%%%%%%%%%%%%%%%%%%%%%%%%%
\subsection*{Role Dynamics Metamodel}
% ALT: Ontological Model of Role Dynamics, Role Dynamics Ontology

The metamodel in \cite{Dastani04} is focused on role dynamics.
Four operations pertaining to the role dynamics are defined: \textit{enact} and \textit{deact} which mean that agent assumes (ALT: acquires) and relinquishes a role respectively, and \textit{activate} and \textit{deactivate} which mean the agent actually starts and stops playing (invoking powers) a role respectively.
Even though it is possible (and very common indeed) for an agent to be enacting multiple roles simultaneously, only one of these can be active at any moment.
Naturally, it is possible that none is active.
In particular, when an agent is invoking (ALT: performing, executing) a power, at that moment exactly one of its roles is active.

%%%%%%%%%%%%%%%%%%%%%%%%%%%%%%%%%%%%%%%%%%%%%%%%%%%%%%%%%%%%%%%%%%%%%%%%%%%%%%%%
\subsection*{Unified Model}

The authors of powerJade merged the metamodels in \cite{Boella04} (organization strucure) and \cite{Boella04} (role dynamics) to model \textit{open systems}\footnote{Open multiagent system is a system that agents can enter and/or leave.}.
Organizations and roles are not just design-time constructions (ALT: abstractions) and players are not isolated agents - the are all agents interaction with one another.
A logical specification of this unified model can be found in \cite{Boella07}.

%%%%%%%%%%%%%%%%%%%%%%%%%%%%%%%%%%%%%%%%%%%%%%%%%%%%%%%%%%%%%%%%%%%%%%%%%%%%%%%%
\subsection*{Powers and Requirements}

% Role (AOPwO) vs. interace (OOP)
Roles can be compared to interfaces from OOP; just like an interface is a contract between a calling class and called class, a role is a contract between an organization and an agent.

% Relationship between interface and implementing entity (class, player agent instance)
In OOP, the relationship between an interface and a class is a static one - the class definition (design-time) specifies a set of interfaces the class implements and no interfaces can be added or removed at run-time.
In contrast, the relationship between a role and a player agent instance in AOP with Organizations (AOPwO) is dynamic - the player agent instance definition (design-time) usually does not specify roles it plays, but it can enact and deact roles at run-time.

% Interface methods & events vs. role requirements & powers
Interfaces in OOP declare methods the implementing class must implement and events it may raise (ALT: fire, trigger).
Interface methods are the class' \textit{responsibilities} (what it \textit{must} do?) and the interface events are the class' \textit{competences} (what it \textit{can} do?).
Similarly, role \textit{requirements} are the responsibilities the player agent takes by enacting/activating the role and the role \textit{powers} are the competences they player gains by enacting/activating the role.
Thus, interface methods correspond to role requirements, while interface events are analogous to role powers.