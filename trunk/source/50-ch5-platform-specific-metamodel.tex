%%%%%%%%%%%%%%%%%%%%%%%%%%%%%%%%%%%%%%%%%%%%%%%%%%%%%%%%%%%%%%%%%%%%%%%%%%%%%%%%
%% Title (en): Multiagent Systems and Organizations                           %%
%% Title (cs): Multiagentní systémy a organizace                              %%
%%                                                                            %%
%% Author: Bc. Lukáš Kúdela                                                   %%
%% Supervisor: Prof. RNDr. Petr Štěpánek, DrSc.                               %%
%%                                                                            %%
%% Academic year: 2011/2012                                                   %%
%%%%%%%%%%%%%%%%%%%%%%%%%%%%%%%%%%%%%%%%%%%%%%%%%%%%%%%%%%%%%%%%%%%%%%%%%%%%%%%%

\chapter{Platform-Specific Metamodel -- Thespian4Jade}

In this chapter, we will describe the platform-

% About platform-specific models 
The platform-specific model (PSM) is a model of a software system bound to a specific technological platform (e.g. a hardware environment (processor), operating system or software environment (virtual machine)).
Platform-specific models are indispensable for the actual implementation of a software system.
% [Wikipedia: http://en.wikipedia.org/wiki/Platform_Specific_Model]

%%%%%%%%%%%%%%%%%%%%%%%%%%%%%%%%%%%%%%%%%%%%%%%%%%%%%%%%%%%%%%%%%%%%%%%%%%%%%%%%
\section{Agent Platform}

% About agent platforms
There are numerous agent platforms available including proprietary, free/open source, and public domain software.

% General vs. specific agent plaforms
The primary distinguishing factor among them is the degree of generality.
The most general among the agent platforms do not impose any particular agent architecture and usually pick a general-purpose programming language (typically the Java programming language) as their AOP language.
On the other hand, the least general (or more generously - most specific) agent platforms do prescribe a concrete agent architecture and generally make use of some (declarative) domain-specific language.

% Why JADE?
As our aim was to introduce organizational concepts as first-class citizens in the MAS landscape, we had to steer towards the agent platform offering the most extension points.
Therefore, we considered only the most general free and open source agent platforms.
Finally we picked the JADE platform (described in the following section) which, at the time of writing this thesis, appears to be the community choice number one.  

%%%%%%%%%%%%%%%%%%%%%%%%%%%%%%%%%%%%%%%%%%%%%%%%%%%%%%%%%%%%%%%%%%%%%%%%%%%%%%%%
\section{Jade}

% About JADE
JADE (Java Agent Development Framework) is an agent platform and framework in one.

% JADE as agent platform
It is a \textit{platform} because it provides a run-time environment in which the agents operate, much like a sandbox in which children play.
JADE also manages the lifetimes of the agents running on the platform, delivers messages (asynchronous message passing) and provides other services (e.g. the yellow pages service).

% JADE as a framework
JADE is also a \textit{framework} because it comes with an extensive base class libraries and the actual user-defined classes are expected to inherit from these base classes, much like a building is expected to be built within the boundaries of an already erected scaffolding.
A concrete multiagent system is basically an instantiation of the JADE as a framework.

% About JADE 2
JADE simplifies the implementation of multiagent systems by acting as a middleware that complies with the FIPA
[FIPA: http://www.fipa.org]
specifications
[FIPA-Specificaiton: http://www.fipa.org/specifications/index.htmls] and by providing a set of graphical tools that support the debugging and deployment phases.
The agent platform can be distributed across multiple machines (possibly running different operating systems) and it can be configured via a remote graphical tool.
The configuration can be even changed at run-time by moving agents from one machine to another, as and when required. 
JADE is written in the Java programming language. 

% Extensibility
JADE has been specifically designed with extensibility in mind and since our goal is to extend the multiagent systems with organizational concepts, this is the platform of our choice.
The fact that it places great emphasis on extensibility also manifests in its choice of the actual AOP language - the Java programming language.

% Minimalistic architecture
A feature of JADE we find particularly noteworthy is its minimalistic architecture.
Only the most fundamental services (e.g. message delivery) are hard-wired.
Whenever possible, a service is implemented as a full-fledged agent operating on the platform to whom messages can be sent (e.g. white and yellow pages service).
Even the graphical tools are only GUI front-ends to these service agents.
This minimalistic architecture is further evidence that JADE takes extensibility seriously.

% License
JADE is a free and open source software, distributed under GNU Lesser General Public Licence (LGPL), version 2.
% TODO Check the spelling of "license".
The copyright holder is Telecom Italia.

% Version 
The version of JADE used in this thesis is JADE 4.1.1 released on November 18th, 2011.

% System requirements
To run JADE 4.1.1, the Java platform, version 1.4 or higher is required.

%%%%%%%%%%%%%%%%%%%%%%%%%%%%%%%%%%%%%%%%%%%%%%%%%%%%%%%%%%%%%%%%%%%%%%%%%%%%%%%%
\section{Thespian4Jade}