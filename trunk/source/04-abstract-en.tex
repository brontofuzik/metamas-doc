%%%%%%%%%%%%%%%%%%%%%%%%%%%%%%%%%%%%%%%%%%%%%%%%%%%%%%%%%%%%%%%%%%%%%%%%%%%%%%%%
%% Title (en): Multiagent Systems and Organizations                           %%
%% Title (cs): Multiagentní systémy a organizace                              %%
%%                                                                            %%
%% Author: Bc. Lukáš Kúdela                                                   %%
%% Supervisor: Prof. RNDr. Petr Štěpánek, DrSc.                               %%
%%                                                                            %%
%% Academic year: 2011/2012                                                   %%
%%%%%%%%%%%%%%%%%%%%%%%%%%%%%%%%%%%%%%%%%%%%%%%%%%%%%%%%%%%%%%%%%%%%%%%%%%%%%%%%

\vbox to 0.5\vsize{

% Disable the paragraph indentation.
\setlength{\parindent}{0mm}
\setlength{\parskip}{5mm}

Thesis title: Multiagent Systems and Organizations\\
Author: Bc. Lukáš Kúdela\\
Author's e-mail address: \url{lukas.kudela@gmail.com}\\
Department: Department of Theoretical Computer Science and Mathematical Logic\\
Thesis supervisor: Prof. RNDr. Petr Štěpánek, DrSc.\\
Supervisor's e-mail address: \url{petr.stepanek@mff.cuni.cz}\\

Abstract: One way to attack a problem is to imagine how a human organization would go about solving it and model this organization as a multiagent system (MAS).
The problem with this approach lies in the simple fact that no single agreed-upon (standard) notion of organization currently exists in the field of MAS.
In practice, this means that every agent can, in principle, talk to/with any other agent regardless of whether this is desirable or even allowed within the modelled organization.
However, this is seldom the case in the real world.
Here all organizations but the simplest ones are usually structured into sub-organizations (branches, divisions, departments, etc.) which can be further decomposed.
They define roles and interaction protocols.
The individual members of the organization assume these roles after meeting declared requirements.
They follow the interaction protocols associated with the role to fulfill its responsibilities.
The aim of this presentation is to demonstrate how the concepts related to organizational structure (oranization, role, player) can be introduced as first-class citizens of MAS (just like the concept of Agent) and present a metamodel that defines their structural and behavioral relationships.

Keywords: multiagent systems, organizations, roles, metamodel

\vss}